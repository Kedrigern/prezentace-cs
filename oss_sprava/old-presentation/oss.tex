\documentclass[12pt]{beamer}

\mode<presentation> {
  \usetheme{Singapore}
	% alternativě: 	
  \setbeamercovered{transparent}
}

\usepackage{fontspec}
\usepackage{xunicode} %Unicode extras!
\usepackage{xltxtra}  %Fixes
\usepackage{polyglossia}
\setmainlanguage{czech}

\usepackage{palatino}

\title{Dočkáme se OSS ve veřejném sektoru?}
\author{Ondřej Profant}
\institute[Piráti]{Česká pirátská strana}
\date{\today}

\begin{document}

\begin{frame}
  \titlepage
  \includegraphics[scale=0.4]{pic/plachta_s_okrajem.png}
\end{frame}

\begin{frame}
  \frametitle{Osnova}
  \tableofcontents
\end{frame}

\section{Rozbor}

\subsection{Co přesně chceme?}
\begin{frame}
 \frametitle{Rozbor -- co přesně chceme?}

	Řekněme si, co vlastně chceme:

 \begin{itemize}
  \item otevřené protokoly $\leftarrow$ 
  \item otevřené datové formáty $\leftarrow$
  \item otevřený zdrojový kód
  \item otevřenost ještě neznamená svobodnost (patenty etc.)
  \begin{itemize}
   \item svobodnost pro uživatele $\leftarrow$
   \item svobodnost nakládání (úpravy)
  \end{itemize}
 \end{itemize}
 Pozn.: $\leftarrow$ značí kritické!
\end{frame}

\subsection{Kde to chceme?}
\begin{frame}
 \frametitle{Rozbor -- kde to chceme?}
 Ve veřejném sektoru, ale co to je?
 \begin{itemize}
  \item státní správa (ministerstva, další úřady, \ldots{})
  \item samosprávné celky (kraje, města, \ldots{})
  \item státem zřizované organizace (školy, policie, hasiči, nemocnice, \ldots{})
  \item další formy veřejného? 
 \end{itemize}
\end{frame}

\subsection{Výhody}
\begin{frame}
 \frametitle{Rozbor -- výhody OSS}
 \begin{itemize}
  \item nekonverguje k~monopolům
  \item zajištění kritických požadavků implicitní (otevřené protokoly, formáty)
  \item typicky:
   \begin{itemize}
    \item přenositelnost
    \item dostupnost
   \end{itemize}
  \item prospěšné široké veřejnosti (rozumné návyky, dostupnost, nediskriminování)
 \end{itemize}
\end{frame}


\subsection{Nevýhody}
\begin{frame}
 \frametitle{Rozbor -- nevýhody}
 \begin{itemize}
  \item výhody jsou dlouhodobé (čili se nemusí projevit hned)
  \item odpovědnost (?!)
 \end{itemize}

Opravdu? Samozřejmě tento problém řeší dodavatel, nikoliv výrobce -- stejně jako v~případě jakýchkoliv jiných zakázek.

Z~pohledu uživatele, či provozovatele opravdu nejsou nevýhody. Distributor či výrobce to může vidět jinak, ale to je jejich problém.
\end{frame}

\section{Stav}
\begin{frame}
 \frametitle{Stav}

Proč má tedy OSS tak malou penetraci?

\begin{itemize}
 \item lobby
 \item strach („konzervativní = správné“)
 \item nekompatibilita / nezdokumentovanost předchozího SW (např. formáty doc, xls, \ldots{})
\end{itemize}
\end{frame}

\section{Příklady}

\begin{frame}
  \frametitle{Příklady -- souhrn}
  Úspěchy:
  \begin{itemize}
   \item Francouzské četnictvo 
   \item Mnichov (LiMux)
   \item mnoho dalších (německých) měst
  \end{itemize}
  Neúspěchy (\textbf{?}): 
  \begin{itemize}
   \item Ostrava-jih (2003--2008)
   \item Vídeň (Wienux, 2005--2009)
   \item ministerstvo zahraničí SRN (2003--2011)
  \end{itemize}
  Připravují přechod: Island
\end{frame}

\subsection{francouzské četnictvo}
\begin{frame}
 \frametitle{Příklady -- francouzské četnictvo}
 \begin{itemize}
  \item Firefox, Thunderbird, OpenOffice, později Ubuntu
  \item 85 000 pracovních stanic
  \item 100 000 zaměstnanců
  \item 4 500 poboček
  \item 2 000 000 Eur roční úspora (50~500~000~Kč k~11.~5.~2012) -- jedná se o~krátkodobý údaj, celková úspora bude řádově vyšší
 \end{itemize}

Zdroj: studie Canonicalu: \texttt{\scriptsize{http://goo.gl/zCocu}} , resp: \texttt{\scriptsize{http://www.canonical.com/sites/default/files/active/Casestudy-GendarmerieNationale.pdf}}
\end{frame}

\subsection{Mnichov}
\begin{frame}
 \frametitle{Příklady -- Mnichov}
 \begin{itemize}
  \item LiMux
  \item Firefox, Thunderbird, OpenOffice, Ubuntu
  \item v~současnosti 9 000 pracovních stanic (o~500 víc než plán!), tento rok dalších 3000 (celkem 15 000)
  \item žádná okamžitá úspora 
  \item přebírají další německá města
 \end{itemize}
\end{frame}

\section{OSS a politika}
\begin{frame}
 \frametitle{OSS a politika}
 Proč je OSS problém prosadit z~politického hlediska?
 \begin{itemize}
  \item absence centralizace (absence silného partnera)
  \item krátkodobá bilance (výhody jsou až dlouhodobé)
  \item status quo (nechceme změnu)
  \item nezájem (IT je příliš nové pro politiky)  
  \item arogance trhu (zbytečná absence podpory)
 \end{itemize}
 Čili pozitiva se stávají největšími negativy.
\end{frame}

\subsection{ČR}
\begin{frame}
 \frametitle{OSS a politika v~ČR}
Vícero českých politických stran prosazuje nasazení OSS -- alespoň dle volebního programu, bohužel k~reálným krokům to má velmi často daleko (koneckonců velmi často to prosazují v~programu, který je distribuován jako *.doc).

\bigskip

Z~reálnějších iniciativ poslední doby je zde pokus Kristýny Kočí. Ta chtěla studii od Red Hatu\footnote{http://www.euro.cz/detail.jsp?id=4783}. 
\end{frame}

%\begin{frame}
% \frametitle{OSS a politika v~ČR 2/2}
% \begin{tabular}
%  Strana 	& Adresa  	& Postoj & url\\
%  \hline
%  CSSD		& cssd.cz 	&	 & \\
%  Kdu-Csl	& kdu.cz  	&	 & \\
%  ODS		& ods.cz	& & \\
%  Pirati	& pirati.cz	& Dlouhodobá podpora & \\
%  Svobodní	& svobodni.cz	& ? & \\
%  TOP09		& top09.cz	& Opatrný & \\
%  Zelení	& zeleni.cz	& Dlouhodobá podpora& \\
%\end{tabular}
%
% Metodika:\\
% \texttt{"open source" inurl:}\\
% \texttt{"oss" inurl" inurl:}
%\end{frame}

\section{Zdroje a Závěr}
\subsection{Zdroje}
\begin{frame}
  \frametitle{Zdroje}
  Konkrétní:
  \begin{itemize}
   \item http://www.h-online.com
   \item http://www.abclinuxu.cz
   \item http://www.linuxexpres.cz	
   \item http://www.canonical.com
   \item hrrp://www.liberix.cz
  \end{itemize}
  Obecné:
  \begin{itemize}
   \item česká media (Euro, \ldots{})
   \item stránky pol. stan v ČR
  \end{itemize}
\end{frame}

\subsection{Závěr}
\begin{frame}
  \frametitle{Závěr}
	Děkuji za pozornost.

	\bigskip
	
	Doplňující otázky?

\bigskip

\bigskip

\scriptsize
Copyleft Ondřej Profant, 2012. Všechna práva vyhlazena. Sdílejte, upravujte a nechte sdílet za stejných podmínek. 

Prezentace v~úplné formě\footnote{i se zdrojovými kódy} na vyžádání emailem: ondrej.profant -at- pirati.cz 
\end{frame}

\end{document}

