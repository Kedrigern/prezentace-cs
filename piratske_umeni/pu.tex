\documentclass[xetex]{beamer}


\mode<presentation> {
  \usetheme{Singapore}
%  \usetheme{Frankfurt}
  \setbeamercovered{transparent}
}

\usepackage{xunicode}
\usepackage{xltxtra}
\usepackage[czech]{babel}

\usepackage{palatino}
\usepackage{graphicx}

\title{Pirátské umění}
\author{Ondřej Profant}
\institute[Piráti]{Česká pirátská strana}
\date{\today}

\begin{document}

\begin{frame}
  \titlepage
  \includegraphics[scale=0.4]{plachta_s_okrajem.png}
\end{frame}

\begin{frame}
  \frametitle{Osnova}
  \tableofcontents
\end{frame}

\section{Definice, aneb o čem je řeč?}

\subsection{Piráti}
\begin{frame}
 \frametitle{Pirátská strana}
 \begin{itemize}
  \item U nás vznikla před dvěma lety, samotné hnutí začíná vznikat na přelomu milenia.

  \item Jedním z pilířů \uv{pirátských myšlenek} je ochrana \uv{pirátského umění}.

  \item Od počátku je zde, až symbiotické, provázání mezi svobodnými umělci a Pirátským hnutím (často přirovnáváno ke vztahu stran Zelených a hnutí Greenpeace).
  \end{itemize}
\end{frame}

\begin{frame}
 \frametitle{Pirátský slovníček}
 \begin{itemize}
  \item \uv{autorský zákon} -- kopírovací monopol (srov. s ang. \emph{copyright})
  \item \uv{ochrana autorů} -- snaha perzekuovat konzumenty (převážně takové, kteří se nemohou bránit)
  \item \uv{v zájmu autorů} -- v zájmu netransparentních korporací a kolektivních správců
  \end{itemize}
\end{frame}

\subsection{Pirátské umění}
\begin{frame}
  \frametitle{Definice \uv{pirátské umění}}
  \begin{block}{Negativní konotace}
    \begin{itemize}
      \item Antikomerční?
      \item Zloději?
    \end{itemize}
  \end{block}
  \begin{block}{Rysy}
    \begin{itemize}
      \item Možnost (alespoň) nekomerčně sdílet
      \item Možnost vytvářet odvozená díla
      \item Žadné cílená degradace kvality (DRM)
    \end{itemize}
  \end{block}
\end{frame}

\section{Platformy}
\subsection{Distribucní}
\begin{frame}
  \frametitle{Platformy}
  \begin{block}{Formy distribuce (výběr)}
  \begin{itemize}
    \item YouTube
    \item Jamendo
    \item Wikipedie
  \end{itemize}
  \end{block}
	\subsection{Financování}
  \begin{block}{Finanční modely (výběr)}
   \begin{itemize}
    \item Dary, ang. \emph{donation}
    \item Flatter, \texttt{http://flattr.com}
    \item Předprodej
    \item Symbióza, např. Ubuntu font
   \end{itemize}
  \end{block}
\end{frame}

\section{Vybrané projekty}
\subsection{Příklady}
\begin{frame}
  \frametitle{Vybrané projekty -- zahraniční}
  \begin{block}{Pioneer One}
	Seriál, pro každý díl samostatná sbírka, zatím 5 dílů
	\texttt{vodo.net/pioneerone}
  \end{block}

  \begin{block}{The Tunnel}
	Film prodávaný po jednotlivých snímcích.
	\texttt{www.thetunnelmovie.net}
  \end{block}

  \begin{block}{Sintel}
	\texttt{www.sintel.org}
  \end{block}

  \begin{block}{The Humble Indie Bundle}
	Balík her nabízený za dobrovolnou cenu, možnost určení peněz, bez DRM, \ldots{}
 \end{block}

\end{frame}

\begin{frame}
  \frametitle{Vybrané projekty -- domácí}
  \begin{block}{Pár pařmenů}
	\texttt{www.bandatrotlu.com}
  \end{block}

  \begin{block}{Open Magazin}
	Časopis, měsíčník, financován dary (6000Kč na jedno číslo)
	\texttt{www.openmagazin.cz}
  \end{block}
\end{frame}
\subsection{Další členové rodiny svobodné kultury}
\begin{frame}
  \frametitle{Další členové rodiny svobodné kultury}
  \begin{itemize}
	\item Svobodný software, ang. \emph{Open Source Software}
		\begin{itemize}
			\item Nejrozsáhlejší, nejpropracovanější
			\item Silné provázání s komerčním sektorem
			\item Např. GNU, Linux, Apache
		\end{itemize}
	\bigskip
	\item Svobodný přístup k informacím, např. Open Acess
		\begin{itemize}
			\item Spíše teorie
			\item Nejsilnější v akadamickém prostředí
		\end{itemize}
  \end{itemize}
\end{frame}

\section{Závěr}
\subsection{Primární}
\begin{frame}
  \frametitle{Zdroje}
  \begin{itemize}
	\item Lawrence Lessig: Svobodná kultura, \texttt{www.root.cz/knihy/svobodna-kultura} res. \texttt{/free-culture}
	\item Richard Stallman: Řada omylů v popleteném copyrightu, 
		\texttt{www.piratopedie.cz/zo:docs:nechape\_copyright}
	\item Eric S. Raymond: Katedrála a tržiště, \texttt{www.root.cz/knihy/katedrala-a-trziste}	
	\item Manifesty jednotlivých umělců
  \end{itemize}
\end{frame}

\subsection{Doplňkové}

\begin{frame}
  \frametitle{Zdroje -- doplňkové}
  \begin{itemize}
	\item Pirátské noviny: \texttt{piratske-noviny.cz}
	\item TorrentFreak: \texttt{torrentfreak.com}
	\item Kinderporno.cz: \texttt{kinderporno.cz}
  \end{itemize}
Většina těchto zdrojů je negativních, popisují spíše negativní dopady kopírovacího monopolu na tvorbu. 
\end{frame}

\begin{frame}
  \frametitle{Závěr}
	Děkuji za pozornost.

	\bigskip
	
	Doplňující otázky?

\bigskip

\bigskip

\scriptsize
Copyleft Ondřej Profant, 2011. Všechna práva vyhlazena. Sdílejte, upravujte a nechte sdílet za stejných podmínek. 

Prezentace v~úplné formě\footnote{i se zdrojovými kódy} na vyžádání emailem: ondrej.profant -at- pirati.cz 
\end{frame}

\begin{frame}
 \frametitle{Odkazy ke stažení ukázek}
 \begin{itemize}
  \item Sintel: http://www.sintel.org/download
  \item Pár pařmenů: http://www.ulozto.cz/10244667/par-parmenu-cd1-avi?utm\_source=search\&utm\_campaign=0\&utm\_medium=all
 \end{itemize}
\end{frame}

\end{document}

