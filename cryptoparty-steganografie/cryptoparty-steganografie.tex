\documentclass[xetex]{beamer}


\mode<presentation> {
%  \usetheme{Singapore}
  \usetheme{Frankfurt}
  \setbeamercovered{transparent}
}

\usepackage{xunicode}
\usepackage{xltxtra}
\usepackage[czech]{babel}
\usepackage{palatino}
\usepackage{graphicx}
\usepackage{tikz}
\usepackage{pgflibraryarrows}
\usepackage{pgflibrarysnakes}

\title{CryptoParty\\Steganografie}

\author{Ondřej Profant}
\institute[Piráti]{Česká pirátská strana}
\date{\today}

\begin{document}

\begin{frame}
  \titlepage
\end{frame}

\begin{frame}
  \frametitle{Osnova}
  \tableofcontents
\end{frame}	

\section{Definice}

\subsection{Steganografie}
\begin{frame}
 \frametitle{Co je Steganografie?}
 \begin{itemize} 
		\item metoda skrytí zprávy
		\item metoda skrytí toho, že zpráva existuje
		\item zašifrování zprávy
 \end{itemize}
\end{frame}

\section{Instalace}
\begin{frame}
 \frametitle{Instalace (vybrané OS)}
	Stránka projektu: \url{http://steghide.sourceforge.net}
 \begin{itemize} 
   \item Ubuntu:
			\begin{enumerate}
				\item \scriptsize{\texttt{wget https://launchpad.net/ubuntu/+archive/primary/+files/steghide\_0.5.1-9build2\_amd64.deb}}
				\item \texttt{sudo dpkg -i steghide\_0.5.1-9build2\_amd64.deb}
			\end{enumerate}
   \item Fedora:\\
			Předpřipravený balík na of. stránkách
   \item Windows:\\
			Předpřipravený balík na of. stránkách
 \end{itemize} 
\end{frame}

\section{Použití}
\begin{frame}
	\frametitle{Použití}
	Připravíme si:
	\begin{itemize}
		\item běžný obrázek: \texttt{obrazek.jpg}
		\item tajnou zprávu (prostý text): \texttt{tajna.zprava.txt}
		\item aplikace: \texttt{steghide}
	\end{itemize}	

Pro zašifrování použijeme příkaz:\\
\texttt{steghide embed\textbackslash{} \\ --embedfile tajna.zprava.txt\textbackslash{} \\ --coverfile obrazek.jpg\textbackslash{} \\ --stegofile obrazek.plus.jpg}

Pro dešifrování použijeme:\\
\texttt{steghide extract\textbackslash{} \\ --stegofile obrazek.plus.jpg\textbackslash{}\\ --extractfile tajna.zprava.txt}
\end{frame}

\section{Závěr}

\begin{frame}
  \frametitle{Závěr}
	Děkuji za pozornost.

	\bigskip
	
	Doplňující otázky?

	\bigskip

	\bigskip

	\scriptsize
	Copyleft Ondřej Profant, 2012. Všechna práva vyhlazena. Sdílejte, upravujte a~nechte sdílet za stejných podmínek. 

	\bigskip

	Prezentace v~úplné formě\footnote{i se zdrojovými kódy} na:\\ 
	\url{https://www.github.com/kedrigern/prezentace-cs}, screencast tvořen v programu Kazam.

	\bigskip

	Mail: ondrej.profant -at- pirati.cz 
\end{frame}

\end{document}
