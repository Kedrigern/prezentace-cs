\documentclass[xetex]{beamer}

\mode<presentation> {
%  \usetheme{Singapore}
  \usetheme{Frankfurt}
  \setbeamercovered{transparent}
}

\usepackage{xunicode}
\usepackage{xltxtra}
\usepackage[czech]{babel}
\usepackage{palatino}
\usepackage{graphicx}
%\usepackage{tikz}
%\usepackage{pgflibraryarrows}
%\usepackage{pgflibrarysnakes}
\usepackage{listings}
\lstset{language=bash,
        numbers=left,
        numberstyle=\tiny,
        showstringspaces=false,
        aboveskip=-40pt,
        frame=leftline
        }

\title{CryptoParty\\Steganografie}

\author{Ondřej Profant}
\institute[Piráti]{Česká pirátská strana}
\date{\today}

\begin{document}

\begin{frame}
  \titlepage
\end{frame}

\begin{frame}
  \frametitle{Osnova}
  \tableofcontents
\end{frame}	

\section{Motivace}
\begin{frame}
 \frametitle{Motivace}

	\begin{block}{Problém}
	V~mnoha státech je šifrování postaveno mimo zákon (resp. je nám k~ničemu).
	\end{block}

Např.:
 \begin{itemize}
    \item<2-6> USA, povinnost osoby dešifrovat, uvažovalo se i~o~povinnosti výrobce šifry dešifrovat *facepalm*
    \item<3-6> Velká Britanie, zákon RIPA
		\item<4-6> Francie, donedávna bylo povoleno šifrovat jen za účelem autentifikace (ověření) etc. V~roce 2004 zrušeno.
		\item<5-6> Pakistán, zákaz VPN
		\item<6> Jistě mnoho dalších autoritářských režimů
 \end{itemize}
\end{frame}


\section{Definice}

\subsection{Steganografie}
\begin{frame}
 \frametitle{Co je Steganografie?}
 \begin{itemize} 
		\item<2-4> metoda skrytí zprávy
		\item<3-4> metoda skrytí toho, že zpráva existuje
		\item<4-4> zašifrování zprávy
 \end{itemize}
\end{frame}

\section{Steghide}

\subsection{Instalace}
\begin{frame}
 \frametitle{Instalace (vybrané OS)}
	Stránka projektu: \url{http://steghide.sourceforge.net}
 \begin{itemize} 
   \item Ubuntu:
			\begin{enumerate}
				\item \scriptsize{\texttt{wget https://launchpad.net/ubuntu/+archive/primary/+files/steghide\_0.5.1-9build2\_amd64.deb}}
				\item \scriptsize{sudo dpkg -i steghide\_0.5.1-9build2\_amd64.deb}
			\end{enumerate}
   \item Fedora:\\
			Předpřipravený balík na of. stránkách
   \item Windows:\\
			Předpřipravený balík na of. stránkách
 \end{itemize} 
\end{frame}

\subsection{Použití}
\begin{frame}
	\frametitle{Použití}
	Připravíme si:
	\begin{itemize}
		\item běžný obrázek: \texttt{obrazek.jpg}
		\item tajnou zprávu (prostý text): \texttt{tajna.zprava.txt}
		\item aplikace: \texttt{steghide}
	\end{itemize}	

Pro zašifrování použijeme příkaz:\\
\scriptsize{steghide embed\textbackslash{} \\ --embedfile tajna.zprava.txt\textbackslash{} \\ --coverfile obrazek.jpg\textbackslash{} \\ --stegofile obrazek.plus.jpg}

Pro dešifrování použijeme:\\
\scriptsize{steghide extract\textbackslash{} \\ --stegofile obrazek.plus.jpg\textbackslash{}\\ --extractfile tajna.zprava.txt}
\end{frame}

\section{Stegotools}
\begin{frame}
	\frametitle{Stegotools}
	Homepage: \url{http://sourceforge.net/projects/stegotools}
	
	Bohužel podporuje pouze bitmapy (*.bmp), což dnes není příliš běžný formát.
\end{frame}

\section{OutGuess}
\begin{frame}
	\frametitle{OutGuess}
	Homepage: \url{http://www.outguess.org}
	
	Obsažen v~linuxových distribucích. 
	
	Šifrování:\\
	\tt{outguess -k heslo -d tajna.zprava.txt obrazek.jpg obrazek.outguess.jpg}
	
	Dešifrování:\\
	\tt{outguess -k heslo -r obrazek.plus2.jpg out.txt}
\end{frame}

\begin{frame}
	\frametitle{Poznámka}
	Programy jsou vzájemně nekompatibilní!
	
	Což je dobře. Jednotný protokol není z~důvodů zachování utajení zprávy.
\end{frame}

\section{Stegdetect}
\begin{frame}
	\frametitle{Stegdetect}
	Program pro zjišťování steganogradie.
	
	Od autora OutGuess.	Mé pokusy nedopadly příliš dobře - určoval velmi nepřesně.
\end{frame}

\section{StegFS}
\begin{frame}
	\frametitle{StegFS}
	Steganografický souborový systém!

	Bohužel, to již někdy příště.

	Homepage: \url{https://albinoloverats.net/projects/stegfs}
\end{frame}


\section{Závěr}

\begin{frame}
	\frametitle{Zdroje}
	
	\begin{enumerate}
	\item homepages jednotlivých projektů
	\item \url{http://cs.wikipedia.org/wiki/Steganografie}
	\item \scriptsize{\url{http://www.root.cz/clanky/jak-ukryt-tajna-data-do-obrazku-aneb-steganografie-v-praxi}}
	\item \scriptsize{\url{http://www.zive.cz/clanky/nejlepsi-program-pro-steganografii/sc-3-a-163982/default.aspx}}
	\item \url{http://en.wikipedia.org/wiki/Cryptography\_law}
	\end{enumerate}
\end{frame}

\begin{frame}
  \frametitle{Závěr}
	Děkuji za pozornost.

	\bigskip
	
	Doplňující otázky?

	\bigskip

	\bigskip

	\scriptsize
	Copyleft Ondřej Profant, 2012. Všechna práva vyhlazena. Sdílejte, upravujte a~nechte sdílet za stejných podmínek. 

	\bigskip

	Prezentace v~úplné formě\footnote{i se zdrojovými kódy} na:\\ 
	\url{https://www.github.com/kedrigern/prezentace-cs}, screencast tvořen v programu Kazam.

	\bigskip

	Mail: ondrej.profant -at- pirati.cz 
\end{frame}

\end{document}
