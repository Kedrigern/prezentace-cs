\documentclass[xetex]{beamer}


\mode<presentation> {
%  \usetheme{Singapore}
  \usetheme{Frankfurt}
  \setbeamercovered{transparent}
}

\usepackage{xunicode}
\usepackage{xltxtra}
\usepackage[czech]{babel}
\usepackage{palatino}
\usepackage{graphicx}
\usepackage{tikz}
\usepackage{pgflibraryarrows}
\usepackage{pgflibrarysnakes}

\title{CryptoParty\\OTR}

\author{Ondřej Profant}
\institute[Piráti]{Česká pirátská strana}
\date{\today}

\begin{document}

\begin{frame}
  \titlepage
\end{frame}

\begin{frame}
  \frametitle{Osnova}
  \tableofcontents
\end{frame}	

\section{Definice}

\subsection{Instant messaging}
\begin{frame}
 \frametitle{Co je IM?}
 \begin{itemize} 
   \item instant messaging
   \item okamžitá komunikace
   \item např. ICQ, Facebook chat, Google talk 
	 \item svobodný prokol jabber
 \end{itemize}
\end{frame}

\subsection{OTR}
\begin{frame}
 \frametitle{Co je OTR?}
 \begin{itemize} 
   \item Off-the-Record Messaging
   \item šifrovací protokol
   \item šifrování probíhá na klientech (koncových uzlech komunikace)
 \end{itemize} 
\end{frame}

\section{Co potřebujeme?}

\begin{frame}
	\frametitle{Co potřebujeme} 

 IM klienta podporujícího OTR.

V našem případě:
	\begin{itemize}
		\item Pidgin - \url{www.pidgin.im}
		\item OTR plugin
		\item Ubuntu 12.10
	\end{itemize}

	Instalace v Ubuntu:\\
	\texttt{sudo apt-get install pidgin pidgin-otr}
\end{frame}

\section{Šifrovaní a autentizace}
\begin{frame}
	\frametitle{Šifrování a autentizace}
	\begin{block}{Šifrování:} zpráva je šifrovaná. To znamená, že si jí lze přečíst pouze se správným klíčem.
	\end{block}

	\begin{block}{Autentizace:} ověření identity. Pokud jen šifrujeme, tak víme pouze to, že na druhé straně je někdo, kdo umí naše zprávy dešifrovat.
	\end{block}

	Více: \url{http://www.cypherpunks.ca/otr/help/3.2.1/levels.php}
\end{frame}

\subsection{Autentizace}
\begin{frame}
	\frametitle{Autentizace}
	3 možnosti ověření identity:
	\begin{itemize}
		\item<1-> Question and answer (otázka a odpověď)
		\item<2-> Shared secret (sdílené heslo)
		\item<3-> Manual fingerprint verification (ověření otisku)
	\end{itemize}
\end{frame}

\section{Praktická ukázka}

\subsection{Ukázka 1}
\begin{frame}
	\frametitle{Ukázka 1 - Pidgin}

	Screencast na youtube:\\ \url{http://youtu.be/4d3pqkelsrU}
\end{frame}

\subsection{Ukázka 2}
\begin{frame}
	\frametitle{Ukázka 2 - Xabber}
	Potřebujeme: Android, Xabber\\
	\url{http://www.xabber.com}\\
	\includegraphics{pic/xabber-android.png}

	Užití velmi podobné jako v Pidginu. 
\end{frame}


\section{Závěr}

\begin{frame}
  \frametitle{Závěr}
	Děkuji za pozornost.

	\bigskip
	
	Doplňující otázky?

	\bigskip

	\bigskip

	\scriptsize
	Copyleft Ondřej Profant, 2012. Všechna práva vyhlazena. Sdílejte, upravujte a~nechte sdílet za stejných podmínek. 

	\bigskip

	Prezentace v~úplné formě\footnote{i se zdrojovými kódy} na:\\ 
	\url{https://www.github.com/kedrigern/prezentace-cs}, screencast tvořen v programu Kazam.

	\bigskip

	Mail: ondrej.profant -at- pirati.cz 
\end{frame}

\end{document}
