\documentclass[xetex]{beamer}

\mode<presentation> {
  \usetheme{Singapore}
%  \usetheme{Frankfurt}
  \setbeamercovered{transparent}
}

\usepackage{xunicode}
\usepackage{xltxtra}
\usepackage[czech]{babel}

\usepackage{palatino}
\usepackage{graphicx}

\usepackage{palatino}
\usepackage{graphicx}

\title{Svobodná kultura}
\author{Ondřej Profant}
\institute[Piráti]{Pirátská strana}
\date{\today}

\begin{document}

\begin{frame}
  \titlepage
  \includegraphics[scale=0.4]{plachta_s_okrajem.png}
\end{frame}

\begin{frame}
  \frametitle{Osnova}
  \tableofcontents
\end{frame}

\section{Definice}

\begin{frame}
	\frametitle{Definice}
	Svobodná kultura je především vymezení proti kultuře nesvobodné.
	
	\begin{enumerate}
		\item Svoboda sdílet
		\item Svoboda modifikovat (parodovat, rozvíjet)
	\end{enumerate}
\end{frame}

\section{Příčiny}
\begin{frame}
	\frametitle{Přirozený proces}
	Snaha o neomezené tvůrčí vyjádření je přirozená.
	
	\bigskip
	
	Příklad: ,,Hollywood'' se přesunul z východního pobřeží na západní kvůli patentům.
	
	\bigskip
	
	Příklad: boj o technologii radia.
\end{frame}

\begin{frame}
	\frametitle{Digitalizace}
	
	Snadná, dostupná:
	\begin{enumerate}
		\item výroba
		\item distribuce
		\item zpětná vazba
	\end{enumerate}
\end{frame}

\begin{frame}
 \frametitle{Zkostnatělé korporace}
Tradiční distributoři, producenti, investoři a jiní zástupci (velko)kapitálu:
 \begin{itemize}
  \item zcela zaspali dobu;
  \item nejsou ochotni inovovat;
  \item nejsou ochotni přistoupit na win-win strategie;
  \item nepochopení, že \uv{globální trh $\neq$ USA+GB+Japonsko};
  \item mají strach;
  \item neumí motivovat lidi.
 \end{itemize}
\end{frame}

\section{Vznik}

\begin{frame}
	\frametitle{Vznik a historie}
	Mnoho tvůrců napříč historii. 
	
	\bigskip
	
	Ve 20. století se objevuje i formalizované hnutí. 
\end{frame}

\begin{frame}
	\frametitle{Hudba}
	Mnoho hudebníků souhlasí, že jim jde o hudbu jako takovou, nikoliv o byznys.
	
	\bigskip
	
	Příklad: Amanda Palmer, Jamendo
	
	\bigskip
	
	Příklad: Pirátský projekt Hrajeme svobodnou hudbu
\end{frame}

\begin{frame}
	\frametitle{Software}
	Svobodný software: sdílení zkušeností a práce. Hnutí založil Richard Stallman. Formuloval základní svobody:
	
	\begin{enumerate}
    \item svoboda používat program za jakýmkoliv účelem
    \item svoboda studovat, jak program pracuje a možnost přizpůsobit ho svým potřebám
    \item svoboda redistribuovat kopie programu.
    \item svoboda vylepšovat program a zveřejňovat zlepšení, aby z nich mohla mít prospěch celá komunita
	\end{enumerate}
	
	\smallskip{}
	
	Dnes plnohodnotná infrastruktura. Od operačních systémů až po animaci.
	
	\smallskip{}
	
	Příklad: Linux, Firefox, LibreOffice
	
	Přiklad distribuce: Humble Bundle, Kickstarter
\end{frame}

\begin{frame}
	\frametitle{Filmy}
	
	Blender Foundation: Blender je svobodný software pro tvorbu 3d objektů, animací etc.
	
	\bigskip
	
	Tvorba titulků.
	
\end{frame}

\section{Konflikt}

\begin{frame}
 \frametitle{Pirátství}
 \begin{itemize}
  \item je přirozená reakce
  \item je doplnění chybějící nabídky
  \item je pohodlnost
  \item je realita
  \item není velký problém
 \end{itemize}
\end{frame}

\section{Zdroje a Závěr}
\subsection{Zdroje}
\begin{frame}
  \frametitle{Zdroje -- obecné}
  \begin{itemize}
	\item Lawrence Lessig: Svobodná kultura, \url{www.root.cz/knihy/svobodna-kultura} res. \texttt{/free-culture}
	\item Richard Stallman: Řada omylů v popleteném copyrightu, 
		\url{www.pirati.cz/zo:docs:nechape\_copyright}
	\item Eric S. Raymond: Katedrála a tržiště, \url{www.root.cz/knihy/katedrala-a-trziste}	
	\item Manifesty jednotlivých umělců
  \end{itemize}
\end{frame}

\subsection{Závěr}
\begin{frame}
  \frametitle{Závěr}
	Děkuji za pozornost.

	\bigskip
	
	Doplňující otázky?

\bigskip

\bigskip

\scriptsize
Copyleft Ondřej Profant, 2011. Všechna práva vyhlazena. Sdílejte, upravujte a nechte sdílet za stejných podmínek. 

Prezentace v úplné formě\footnote{i se zdrojovými kódy} na: \url{https://github.com/Kedrigern/prezentace-cs}
\end{frame}


\end{document}

