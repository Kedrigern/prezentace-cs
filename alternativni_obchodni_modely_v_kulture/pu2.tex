\documentclass[xetex]{beamer}


\mode<presentation> {
  \usetheme{Singapore}
%  \usetheme{Frankfurt}
  \setbeamercovered{transparent}
}

\usepackage{xunicode}
\usepackage{xltxtra}
\usepackage[czech]{babel}

\usepackage{palatino}
\usepackage{graphicx}

\usepackage{palatino}
\usepackage{graphicx}

\title{Alternativní obchodní modely v kultuře\\
aneb\\
Jak vydělat a přitom být slušný}
\author{Ondřej Profant}
\institute[Piráti]{Česká pirátská strana}
\date{\today}

\begin{document}

\begin{frame}
  \titlepage
  \includegraphics[scale=0.4]{plachta_s_okrajem.png}
\end{frame}

\begin{frame}
  \frametitle{Osnova}
  \tableofcontents
\end{frame}

\section{Příčina problému}

\subsection{Zkostnatělé korporace}
\begin{frame}
 \frametitle{Zkostnatělé korporace}
Tradiční distributoři, producenti, investoři a jiní zástupci (velko)kapitálu:
 \begin{itemize}
  \item zcela zaspali dobu;
  \item nejsou ochotni inovovat;
  \item nejsou ochotni přistoupit na win-win strategie;
  \item nepochopení, že \uv{globální trh $\neq$ USA+GB+Japonsko};
  \item mají strach.
 \end{itemize}

Samozřejmě najdeme výjimky. O těch budeme mluvit později.
\end{frame}

\subsection{Pirátsví}
\begin{frame}
 \frametitle{Pirátství}
 \begin{itemize}
  \item je přirozená reakce
  \item je doplnění chybějící nabídky
  \item je pohodlnost
  \item je realita
  \item není velký problém
 \end{itemize}
\end{frame}

\begin{frame}
 \frametitle{Je pirátství problém?}
 \begin{itemize}
  \item jedná se o formu reklamy 
  \item často se setkáváme s nepravdivými tezemi: 
	\begin{itemize}
	\item \uv{upirátil = nekoupil}
 	\item \uv{kdyby nepirátil, koupi by}
    \item \uv{piráti nechtějí platit}
	\end{itemize}
  \item nezanedbatelná část pirátů je piráty, protože neexistuje přiměřená nabídka
 \end{itemize}
\end{frame}

\section{Platformy a modely}
\subsection{Úvod}

\begin{frame}
\frametitle{Disclaimer}
Není v mým silách zachytit všechny formy alternativní distribuce.

Stejně tak naše časové možnosti jsou omezené.

Proto další část prezentace berte jako informativní průřez, nikoliv vyčerpávající výčet.
\end{frame}

\begin{frame}
\frametitle{Obecná pravidla (1/2)}

	\begin{enumerate}
	\item Prospěšné je část tvorby zveřejnit bez omezení
		\begin{itemize}
		\item Je třeba mnoho reklamy 
		\item Nejlepší reklamou jsou hotová díla 
		\end{itemize}
	\item Je zbytečné tvorbu mrzačit různými DRM
		\begin{itemize}
		\item Nezabírá
		\item Zpravidla postihuje platící zákazníky
		\item O'Reilly zaznamenal 104\% nárůst prodeje e-knih po zrušení DRM
		\end{itemize}	

	\end{enumerate}
\end{frame}

\begin{frame}
\frametitle{Obecná pravdila (2/2)}
	
	\begin{enumerate}
		\setcounter{enumi}{2}
		\item Je třeba srazit náklady distribuce a výběru odměny
			\begin{itemize}
			\item Problém žáby na prameni
			\item Technologie jsou
			\item \ldots{}
			\end{itemize}
		\item Doplňkové služby
			\begin{itemize}
			\item Merchandising
			\item Sběratelské edice
			\item Autogramiády
			\item \ldots{}
			\end{itemize}
		\item Neurážet zákazníky
	\end{enumerate}

\end{frame}

\begin{frame}
\frametitle{Malířství a klasické formy umění}

\begin{block}{Používají již dlouho}
	Originál obrazu (i jiných uměleckých děl) je výrazně (o~několik řádů) dražší, než jeho téměř identická kopie. Přesto je to tržní cena.
\end{block}
\end{frame}

\begin{frame}
\frametitle{Platformy -- výběr}  

\begin{block}{Formy distribuce (komerčnější)}
  \begin{itemize}
    \item iTunes 
 	\item Google music
    \item hulu.com	-- seriály
	\item \ldots{}
  \end{itemize}
  \end{block}


\begin{block}{Formy distribuce (svobodnější)}
  \begin{itemize}
    \item YouTube -- video, hudba
 	\item VODO -- video
    \item Jamendo -- hudba
	\item \ldots{}
  \end{itemize}
  \end{block}
\end{frame}

\subsection{Financování}
\begin{frame}
 
 \begin{block}{Alternativní finanční modely (výběr)}
	   \begin{itemize}
	    \item Dary, ang. \emph{donation}
	    \item Flatter, \texttt{http://flattr.com}
			\begin{itemize}
				\item Koncept mikroplateb
			\end{itemize}
	    \item Předprodej
			\begin{itemize}
				\item Uvidíme později
			\end{itemize}
	    \item Symbióza, např. Ubuntu font
			\begin{itemize}
				\item Vydělávám jen na vybraných službách, ale poskytuji široký ekosystém
			\end{itemize}
   		\end{itemize}
  \end{block}
\end{frame}

\section{Vybrané projekty}
\subsection{VODO}
\begin{frame}
  \frametitle{VODO}
	\begin{itemize}
		\item Alternativní distribuce a platby za filmy a seriály
		\item Celkem 141 projektů	
		\item K distribuci využivá BitTorrent
		\item U každého projektu je možnost přispět -- nelze brát úspěch dle klasických měřítek, některé projekty mohou stabilně vydělávat ještě dlouho
	\end{itemize}
\end{frame}

\begin{frame}
\frametitle{VODO -- Projekty}
  \begin{block}{Pioneer One}
	Seriál, pro každý díl samostatná sbírka, zatím 6 dílů
	Celkově vybráno: US \$ 89 944 (1 663 964 Kč)
  \end{block}

  \begin{block}{The Tunnel}
	Film prodávaný po jednotlivých snímcích.
	Prodali se snímky za AUD \$ 36 000 (716 400 Kč)
  \end{block}

  \begin{block}{L5}
	Jeden z nejnovějších projektů.\\
	Pustíme si ukázku (2 minuty)
  \end{block}
\end{frame}

\subsection{The Humble Indie Bundle}

\begin{frame}
\frametitle{The Humble Indie Bundle (1/2)}
	\begin{itemize}
		\item Serie jednorázových akcí
		\item V balíku je vždy několik her
		\item Zakázník zaplatí libovolnou částku
		\item Zákazník rozhoduje jak se jeho platba rozdělí
	\end{itemize}

Dané hry:
	\begin{itemize}
		\item Neobsahují DRM
		\item Jsou multiplatformní 
		\item (Jsou zábavné :-) )
	\end{itemize}
\end{frame}

\begin{frame}
\frametitle{The Humble Indie Bundle (2/2)}

\begin{center}
\footnotesize{
\begin{tabular}{|l |c | r |}
\hline
název													&rok		&tržby\\
\hline
Humble Indie Bundle 1 					&2010		&1.27+ M\\
Humble Indie Bundle 2 					&2010		&1.8+ M\\
Humble Frozenbyte Bundle			&2011		&0.9+ M\\
Humble Indie Bundle 3					&2011		&2.16+ M\\
Humble Frozen Synapse Bundle	&2011		&1.11+ M\\
Humble Voxatron Debut					&2011		&0.9+ M\\
Humble Introversion Bundle			&2011		&0.77+ M\\
Humble Indie Bundle 4					&2011		&2.37+ M\\
Humble Bundle for Android			&2012		&0.93+ M\\
Humble Bundle Mojam					&2012		&0.45+ M\\
\hline
Celkem												&3 roky	&12.66+ M\\
\hline
\end{tabular}}

\medskip

\small{Toto je zisk pouze z HIB, dané hry se typicky prodávají na dalších místech.}
\end{center}
\end{frame}

\subsection{Open source}
\begin{frame}
  \frametitle{Open source software (OSS)}
Postavení:
	\begin{itemize}
		\item Jedna z nejstarších alternativních metod
		\item Má i ekonomickou sílu
		\item Dnes tento model respektují i velcí hráči na trhu
	\end{itemize}
Zástupci:
	\begin{itemize}
		\item Red Hat
			\begin{itemize}
				\item Platí se za podporu
				\item Soustavný stabilní ekonomický růst
				\item Čistý zisk za Q3 2011: 38,2 milionů \$\\ (706.7 milionů Kč)
			\end{itemize}
		\item SUSE
	\end{itemize}
\end{frame}

\section{Zdroje a Závěr}
\subsection{Zdroje}
\begin{frame}
  \frametitle{Zdroje -- obecné}
  \begin{itemize}
	\item Lawrence Lessig: Svobodná kultura, \texttt{www.root.cz/knihy/svobodna-kultura} res. \texttt{/free-culture}
	\item Richard Stallman: Řada omylů v popleteném copyrightu, 
		\texttt{www.piratopedie.cz/zo:docs:nechape\_copyright}
	\item Eric S. Raymond: Katedrála a tržiště, \texttt{www.root.cz/knihy/katedrala-a-trziste}	
	\item Manifesty jednotlivých umělců
  \end{itemize}
\end{frame}


\begin{frame}
  \frametitle{Zdroje -- konkrétních projektů}
  \begin{itemize}
	\item VODO: \texttt{cs.wikipedia.org/wiki/VODO}
	\item The Humble Indie Bundle: \texttt{/www.humblebundle.com}
	\item Red Hat tržby: \tiny{\texttt{channelworld.cz/software/\\
red-hat-zverejnil-financni-vysledky-za-treti-financni-ctvrtleti-2012-5461}}
  \end{itemize}
\end{frame}

\begin{frame}
  \frametitle{Zdroje -- doplňkové}
  \begin{itemize}
	\item Pirátské noviny: \texttt{piratske-noviny.cz}
	\item TorrentFreak: \texttt{torrentfreak.com}
	\item Kinderporno.cz: \texttt{kinderporno.cz}
  \end{itemize}
Většina těchto zdrojů je negativních, popisují spíše negativní dopady kopírovacího monopolu na tvorbu. 
\end{frame}

\subsection{Závěr}
\begin{frame}
  \frametitle{Závěr}
	Děkuji za pozornost.

	\bigskip
	
	Doplňující otázky?

\bigskip

\bigskip

\scriptsize
Copyleft Ondřej Profant, 2011. Všechna práva vyhlazena. Sdílejte, upravujte a nechte sdílet za stejných podmínek. 

Prezentace v úplné formě\footnote{i se zdrojovými kódy} na vyžádání emailem: ondrej.profant -at- pirati.cz 
\end{frame}


\end{document}

